%2.	Background
%Background information needed to understand the technical details in the rest of the report
\section{Method}

The sales history dataset provided contained the sale history of 60 different Russian shops. It contained daily sales of over 22 000 different items over the time of 33 months. Totaling about 3 million listings. Features also included was date, item price and item category.
\newline
The data that was provided was in the form of large \emph{csv} documents. Most of these contained only supplemental information about the names of items, categories and shop.all the data used for training was in \empth{salesTrain}. The Python library \emph{pandas} was used to read this large \emph{csv} file.
\newline
For a given month and a given store could have sold the same item more than once. Therefore the data was grouped by shop, month, and item. This way each item had only one data point for each shop and month.
\newline
The covariance matrix of the data was calculated. This was used to make sure the features was not strongly dependent on each other. Each variable should not be determinable from other variables in the data.
\newline
The data was split into a training and a testing data.
\newline
The performance of a given regression model was measured using R-squared on the testing dataset. R-squared calculates the proportion of the variation in the test data that is explained by the regression (Eq \ref{eq:R2}). If R-squared is 1, all of the variation in the test data could be explained by the regression model.

\begin{equation}
    R^2 = 1-\frac{Var(\text{Unexplained})}{Var(\text{Explained})}
    \label{eq:R2}
\end{equation}
\newline
The R-squared seen in the results is the adjusted R-squared (Eq \ref{eq:R2a}). It is adjusted for the number of data points ($n$) and the number of regressors ($k$). Unlike the normal R-squared, the value can be negative when the $R^2$ term is close to 0 meaning the model is very inaccurate.

\begin{equation}
R_{a d j}^{2}=1-\frac{\left(1-R^{2}\right)(n-1)}{n-k-1}
\label{eq:R2a}
\end{equation}





